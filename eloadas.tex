\documentclass{beamer} % bemutatóhoz: \documentclass{beamer}
\usepackage{t1enc}
\def\magyarOptions{defaults=hu-min}
\usepackage[magyar]{babel}
\usepackage{ragged2e}
\let\raggedright=\RaggedRight
\usepackage[utf8]{inputenc}
\usepackage{amsmath}
\usepackage{amsfonts}
\usepackage{amssymb}
\usepackage{amsthm}
\usepackage{mathrsfs}
\usepackage{mathptm}
\usepackage{physics}
\usepackage{times}
\newtheorem{lem}{Lemma}[section]
\newtheorem{theo}[lem]{Tétel}
\newtheorem{defi}[lem]{Definíció}
\newtheorem{megall}[lem]{Megállapodás}
\newtheorem{allitas}[lem]{Állítás}
\newtheorem{atfog}[lem]{Átfogalmazás}
\DeclareMathOperator{\dom}{dom}
\DeclareMathOperator{\ess}{ess}
\usetheme{Madrid}
\setbeamertemplate{footline}{}
\setbeamertemplate{navigation symbols}{}
\title{Banach-tér értékű integrálok és a Radon\---Nikodym tulajdonság}
\subtitle{alkalmazott matematikus BSc. szakdolgozat}
\author{Seregi Benjámin Martin}
\institute{Eötvös Loránd Tudományegyetem}
\date{2016}
\begin{document}
\frame{\titlepage}
\begin{frame}
\frametitle{Célkitűzés (Section 1)}
\justifying
\begin{itemize}
\item bevezető a Banach-tér értékű integrálelméletbe
\pause \item meglévő mértékelméleti fogalmak általánosítása
\end{itemize}
\pause \begin{megall}
A továbbiakban $\mathcal{X}$ egy Banach-teret és $(\Omega, \mathcal{A}, \mu)$ egy véges mértékteret jelöl. Az $\mathcal{X}^*$ az $\mathcal{X}$ duális terét jelöli és egy elemére $x^*$-al hivatkozunk.
\end{megall}
\pause \begin{defi}
Banach-tér értékű függvény alatt egy  $f \colon \Omega \to \mathcal{X}$ függvényre gondolunk.
\end{defi}
\end{frame}

\begin{frame}
\frametitle{Mérhetőség (Section 2)}
\justifying
\textbf{Hogyan lehet a mérhetőség fogalmát bevezetni Banach-tér értékű függvényekre?}
\begin{itemize}
\pause \item gyenge mérhetőség
\pause \item erős mérhetőség (röviden: mérhető)
\end{itemize}
\end{frame}

\begin{frame}
Egyszerű függvények:
$$f(\omega) = \sum^k_{i=1} \chi_{A_i}(\omega)x_i \quad (x_i \in \mathcal{X}).$$
\pause \begin{defi}[erős mérhetőség] Egy $f \colon \Omega \to \mathcal{X}$ függvény erősen mérhető, ha létezik $(f_n)_{n \in \mathbb{N}}$ egyszerű függvényeknek olyan sorozata, hogy $f_n(\omega) \to f(\omega)$ $\mu$-majdnem minden $\omega \in \Omega$-ra.
\end{defi}
\pause\begin{defi}[gyenge mérhetőség] Egy $f \colon \Omega \to \mathcal{X}$ gyengén mérhető, ha bármely $x^* \in \mathcal{X}^*$ folytonos lineáris funkcionálra az $\omega \mapsto x^*(f(\omega))$ függvény mérhető.
\end{defi}
\end{frame}

\begin{frame}
\justifying
\begin{itemize}
\item a definíciók és az elnevezések jók
\end{itemize}
\pause 
\begin{theo}[Pettis mérhetőségi tétele, Theorem 2.9] Legyen $(\Omega,\mathcal{A},\mu)$ egy $\sigma$-véges mértéktér. Egy $f \colon \Omega \to \mathcal{X}$ függvény pontosan akkor erősen mérhető, ha:

a) gyengén mérhető és

b) majdnem mindenütt szeparábilis értékkészletű.
\end{theo}
\begin{itemize}
\pause \item szeparábilis terekben ugyanazt jelentik
\end{itemize}
\end{frame}

\begin{frame}
\frametitle{Integrálhatóság és az integrálok tulajdonságai (Section 3)}
\textbf{Hogyan lehet egy mérhető függvényt integrálni?}
\begin{itemize}
\pause \item mérhető függvény $\leadsto$ Pettis-integrál vagy Bochner-integrál
\pause \item gyengén mérhető függvény $\leadsto$ Pettis-integrál
\end{itemize}
\end{frame}

\begin{frame}
\justifying
\begin{defi}[Bochner-integrál] Legyen $f \colon \Omega \to \mathcal{X}$ egy $f(\omega) = \sum^{k}_{i=1} \chi_{A_i}(\omega) x_i$ alakú egyszerű függvény. Ekkor $f$ Bochner-integrálja $\Omega$-n a $\mu$ mérték szerint:
$$\int_{\Omega} f \dd{\mu} := \sum^{k}_{i=1} \mu(A_i) x_i.$$
Ha $f \colon \Omega \to \mathcal{X}$ egy tetszőleges erősen mérhető függvény és $(f_n)_{n \in \mathbb{N}}$ egy hozzá $\mu$-majdnem minden pontban konvergáló egyszerű függvénysorozat, amely teljesíti a következő feltételt:
$$\lim_{n \to \infty}\int_{\Omega} \| f_n(\omega) - f(\omega) \| \dd{\mu} = 0,$$
akkor $f$ Bochner-integrálja az $\Omega$ halmazon a $\mu$ mérték szerint:
$$\int_{\Omega} f \dd{\mu} := \lim_{n \to \infty} \int_{\Omega} f_n \dd{\mu}.$$
\end{defi}
\end{frame}

\begin{frame}
\justifying
\begin{theo}[Bochner, Theorem 3.4] Egy erősen mérhető $f \colon \Omega \to \mathcal{X}$ függvény pontosan akkor Bochner-integrálható, ha
$$\int_{\Omega} \| f(\omega) \| \dd{\mu} < \infty.$$
\end{theo}
\begin{itemize}
\pause \item $\left \| \int_{\Omega} f \dd{\mu} \right\| \leqslant \int_{\Omega} \| f(\omega) \| \dd{\mu}$
\pause \item $\nu_f(A) := \int_{A} f \dd{\mu}$ $\sigma$-additív.
\end{itemize}
\end{frame}

\begin{frame}
\justifying
\begin{theo}[Hille, Theorem 3.13] Legyen $\mathcal{X}_1$ és $\mathcal{X}_2$ két Banach-tér, valamint $L \colon \mathcal{X}_1 \to \mathcal{X}_2$ egy zárt lineáris operátor. Ha $f \colon \Omega \to \dom L$ és $L(f(\omega))$ függvények Bochner-integrálhatóak, akkor
$$L\left(\int_{\Omega} f \dd{\mu} \right) = \int_{\Omega} L(f(\omega)) \dd{\mu}.$$
\end{theo}
\begin{itemize}
\pause \item minden $A \in \mathcal{A}$-ra: $\int_{A} f \dd{\mu} = \int_{A} g \dd{\mu} \Longrightarrow f = g$ m. m.
\pause \item Bochner-tér
\end{itemize}

\pause Bochner-tér normája

\begin{displaymath}
 \| f \|_{L^p} := \left\{
    \begin{array}{ll}
      \left(\int_{\Omega} \| f(\omega) \|^p  \dd{\mu} \right)^{\frac{1}{p}}, &  1 \leqslant p < \infty \\
      \ess\sup_{\Omega} \|f(\omega)\| = \inf \lbrace C \geqslant 0 : \|f(\omega)\| \leqslant C \text{ $\mu$-m.m.} \rbrace, &  p = \infty.
    \end{array}
  \right.
\end{displaymath}
\end{frame}

\begin{frame} 
\justifying
\begin{itemize}
\item $L^p(\mathcal{X}; \Omega, \mathcal{A}, \mu)$ függvények ekvivalenciaosztályai + norma
\pause \item $L^p(\mathcal{X}; \Omega, \mathcal{A}, \mu)$ valójában Banach-tér
\end{itemize}
\end{frame}

\begin{frame}
\justifying
\begin{defi}[skalárisan- és Pettis-integrálható függvény] Egy gyengén mérhető $f \colon \Omega \to \mathcal{X}$ függvény skalárisan integrálható, ha bármely $x^* \in \mathcal{X}^*$-ra az $\omega \mapsto x^*(f(\omega))$ függvény $L^1(\Omega, \mathcal{A},\mu)$-beli. Egy skalárisan integrálható $f$ függvény Pettis-integrálható, ha létezik egy $x \in \mathcal{X}$ úgy, hogy bármely $x^* \in \mathcal{X}^*$-ra
$$x^*(x) = \int_{\Omega} x^*(f(\omega))\dd{\mu}.$$
Ekkor az $f$ Pettis-integrálja $x$.
\end{defi}
\begin{itemize}
\pause \item Bochner-integrál ''erős'', Pettis-integrál ''gyenge''
\pause \item integrálható függvények véges és végtelen dimenzióban
\end{itemize}
\end{frame}

\begin{frame}
\justifying
\begin{itemize}
\item nincs jó becslés
\pause \item $\sigma$-additivitás megmarad
\pause \item nincs könnyen ellenőrizhető feltétel az integrálhatóságra
\pause \item minden $A \in \mathcal{A}$-ra: $\int_{A} f \dd{\mu} = \int_{A} g \dd{\mu} \Longrightarrow f = g$ m. m.
\end{itemize}
\pause Pettis-tér = integrálható függvények ekvivalenciaosztályai + norma.

\pause Norma a téren 
$$
\|f\|_{\mathcal{P}(\mathcal{X})} := \sup_{x^* \in \mathcal{X}^*} \int_{\Omega} | x^*(f(\omega)) | \dd{\mu}.
$$
\pause Ez már nem mindig Banach-tér.
\end{frame}

\begin{frame}
\justifying
\frametitle{Vektormértékek (Section 4)}
\textbf{Hogyan lehet a mérték fogalmát általánosítani?}
\pause \begin{defi}[vektormérték]Vektormértéknek egy $\sigma$-additív  $F \colon \mathcal{A} \to \mathcal{X}$ halmazfüggvényt nevezünk. Egy vektormérték abszolút folytonos $(F \ll \mu)$ a $\mu$ mértékre nézve, ha teljesül a következő: ha egy $A \in \mathcal{A}$ halmazra $\mu(A) = 0$, akkor $F(A)=0$ is teljesül.
\end{defi}
\pause A határozatlan Bochner- és Pettis-integrál:
$$\nu_f(A) := \int_{A} f \dd{\mu} \quad (A \in \mathcal{A})$$
\end{frame}

\begin{frame}
\justifying
\begin{defi}[variáció]
Legyen $F\colon \mathcal{A} \to \mathcal{X}$ egy vektormérték. Ekkor $F$ variációja
$$\sup_{\pi} \left\lbrace \sum^{m}_{j=1} \| F(A_j) \| : \bigcup^{m}_{j=1} A_j = A \text{ és } \pi := \lbrace A_1, \ldots, A_m \rbrace \text{ diszjunktak} \right\rbrace,$$
amit $|F|(A)$-val jelölünk.
\end{defi}
\begin{itemize}
\pause \item korlátos változású, ha $|F|(\Omega) < \infty$
\pause \item $|F|$ mérték
\end{itemize}
\pause \begin{allitas}[Proposition 4.6] Legyen $f \colon \Omega \to \mathcal{X}$ egy Bochner-integrálható függvény és $F$ az $f$ által generált vektormérték (azaz $\nu_f$, a határozatlan Bochner-integrál). Ekkor 
$$|F|(A) = \int_{A} \| f(\omega) \| \dd{\mu} \quad (A \in \mathcal{A}).$$
\end{allitas}
\end{frame}

\begin{frame}
\justifying
\begin{allitas}[Proposition 4.11] Legyen $F$ egy korlátos változású vektormérték, amely abszolút folytonos a $\mu$ mértékre nézve. Ekkor $|F|$ is abszolút folytonos $\mu$-re nézve.
\end{allitas}
\pause Ezért használhatjuk a Radon\---Nikodym tételt és rögtön kapjuk, hogy van olyan $g \colon \Omega \to [0,\infty]$ mérhető függvény, hogy
$$|F|(A) = \int_{A} g \dd{\mu} \quad (A \in \mathcal{A}).$$ \pause Ha $F(A) = \int_{A} f \dd{|F|}$ valamilyen Bochner-integrálható $f$ függvényre, akkor
\pause $$\int_{A} f \frac{\dd{|F|}}{\dd{\mu}} \dd{\mu} = \int_{A} f \dd{|F|} = F(A).$$
\pause Tehát 
$$\int_{A} fg \dd{\mu} = F(A).$$
\end{frame}

\begin{frame}
\frametitle{A Radon\---Nikodym tulajdonság és a Riesz reprezentálható operátorok (Section 5)}
\justifying
\textbf{Igaz marad-e a Radon\---Nikodym tétel átfogalmazása?}
\pause \begin{atfog}[Radon\---Nikodym tétel] Legyen $F$ egy korlátos változású vektormérték, amely abszolút folytonos $\mu$-re nézve. Ekkor létezik egy olyan Bochner integrálható $f$ függvény, amelyre a következő teljesül:
$$\int_{A} f \dd{\mu} = F(A) \quad (A \in \mathcal{A}).$$
\end{atfog}
\end{frame}

\begin{frame}
\justifying
\begin{itemize}
\item Radon\---Nikodym tétel átfogalmazása általánosan \textbf{nem} igaz
\pause \item Radon\---Nikodym tulajdonság
\end{itemize}
\pause \begin{defi}[Riesz reprezentálható operátor] Egy $T \colon L^1(\Omega, \mathcal{A}, \mu) \to \mathcal{X}$ operátor (Riesz) reprezentálható, hogy ha létezik egy olyan $f\colon \Omega \to \mathcal{X}$ lényegében korlátos Bochner integrálható függvény, hogy
$$T(f) = \int_{\Omega} fg \dd{\mu} \quad(f \in L^1(\Omega, \mathcal{A}, \mu))$$
\end{defi}
\begin{itemize}
\item Radon\---Nikodym tulajdonság kapcsolata $L^1(\Omega, \mathcal{A}, \mu) \to \mathcal{X}$ típusú folytonos operátorokkal
\end{itemize}
\end{frame}
\begin{frame}
\justifying
\begin{theo}[Theorem 5.2] Legyen $\mathcal{X}$ és $(\Omega, \mathcal{A}, \mu)$ egy véges mértéktér. Ekkor $\mathcal{X}$ Radon\---Nikodym tulajdonságú a $(\Omega, \mathcal{A}, \mu)$ mértéktérre nézve akkor és csak akkor, ha bármely folytonos $L^1(\Omega, \mathcal{A}, \mu) \to \mathcal{X}$ típusú operátor reprezentálható.
\end{theo}
\begin{itemize}
\pause \item $c_0$ nem Radon\---Nikodym tulajdonságú
\pause \item Hilbert-terek igen (Dunford-Pettis tétel következménye)
\end{itemize}
\end{frame}
\begin{frame}
\center
\textbf{Köszönöm a figyelmet!}
\end{frame}
\end{document}